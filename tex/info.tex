\begin{center}
  {\sffamily Copyright \faCopyright[regular] 2022, Ícaro Vidal Freire }
  \begin{tblr}{ll}
    \faEnvelope[regular] & \textsf{icarofreire@ufrb.edu.br} \\ 
    \faGithub            & \href{github.com/icaro-freire}{\sffamily github.com/icaro-freire}
  \end{tblr}

\vfill


\begin{tcolorbox}[
    colback=blue!5!white,
    colframe=blue!75!black,
    title=\large\faInfoCircle\quad\textbs{Informações Relevantes}
]
  \footnotesize
  \begin{itemize}
    \item[\faCreativeCommons] Este obra está licenciada com uma 
         \href
         {
           https://creativecommons.org/licenses/by-nc/4.0/deed.pt_BR
         }
         {
         Licença Creative Commons Atribuição-NãoComercial 4.0 Internacional
         }.
         Você, basicamente, pode \textit{compartilhar} ou \textit{adaptar} esse 
         trabalho, desde que o uso seja \textit{não comercial}; e, que 
         \textit{atribua} os devidos créditos.
    \item[\faTerminal] Texto produzido em {\grega \hologo{LaTeX2e}}, 
         com o editor \href{https://neovim.io/}{\texttt{Neovim}} e a ferramenta 
         de automação \href{https://islandoftex.gitlab.io/arara/}{\arara arara}; 
       \item[\faFilePdf] Texto em pdf com dimensão para papel A5;
    \item[\faLinux] Distribuição \href{https://miktex.org/}{\hologo{MiKTeX}}, 
         num sistema \href{https://pop.system76.com/}{\textsf{Pop!\_OS 20.04 LTS}}
    \item[\faPenNib] As fontes utilizadas nesse texto foram:
      
         \begin{tblr}{ll} 
           Alegreya               & corpo do texto (serifada) \\
           \textsf{Alegreya Sans} & destaques ou títulos (\textsf{sem serifa}) \\
           \texttt{Ubuntu Mono}   & códigos \& Cia (\texttt{monoespaçada}) \\
           $eulervm$              & expressões matemáticas ($\Delta = b^2 - 4ac$) \\
           {\grega FreeSerif}     & palavras gregas 
                                    (
                                      \href
                                      {
                                        https://pt.wikipedia.org/wiki/Ichthys
                                      }
                                      {
                                        \grega ἰχθύς
                                      }
                                    ) \\
           {\fontins Fontin Sans} & usada na capa (\,\fontins Tangenciando o \ldots)\\
           {\arara Comfortaa}     & para escrever o nome {\arara arara}
         \end{tblr}

  \end{itemize}

\end{tcolorbox}


\end{center}
