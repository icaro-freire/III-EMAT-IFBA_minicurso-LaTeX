%==========================================================================
% Capítulo 2: Organizando para Começar 
%==========================================================================
\chapter{Organizando para Começar} 
\label{cap:organizando} 


Parte desse capítulo poderia ser opcional, mas acredito que, depois de um tempo 
usando o \LaTeXX, você começará a pensar sobre \textit{organização}. 

E existem muitas vertentes do que seja ``organização'': podemos falar em como 
\textit{escrever} o nome dos arquivos que vamos usar para escrever em \LaTeXX;
podemos falar na organização do diretório onde faremos a composição tipográfica
(colocaremos tudo em um único diretório, ou vamos separar por categorias os 
arquivos?); podemos também pensar na organização da escrita do texto (vamos 
escrever tudo em um único arquivo, ou poderíamos escrever partes do texto em 
arquivos diferentes e, depois, juntarmos tudo num arquivo principal?); etc. 

É sobre isso que falaremos nesse capítulo. 
Mas, desde já, confesso que tudo será bem pessoal. 
Você pode fazer como desejar!

\section{Como escrever os nomes do arquivo para compilação} %---------------

Os compiladores são suficientemente inteligentes para identificar a linguagem 
{ \grega \hologo{LaTeX2e} } em um arquivo de texto. 
Mas, é uma boa prática salvar os arquivos no formato \texttt{.tex}. 
Isso facilita a identificação de ícones por parte das IDEs, por exemplo.

Também são boas práticas: salvar os arquivos \texttt{.tex} com nomes 
\textit{sem acentuação} ou \textit{caracteres especiais do teclado} (\%, \$, \&, etc.). 
E, se o nome do arquivo for composto por mais de uma palavra, é aconselhável 
também \textit{não deixar espaço entre elas}.

Por exemplo, suponha que você esteja escrevendo um \textit{artigo} sobre 
\textit{números complexos}. \textbs{Não} escreva assim: 

\tcboxC{
  Artigo! Números Complexos.tex
}

Seria aconselhável retirar a \textit{exclamação}, o \textit{acento agudo} e os 
\textit{espaços entre as palavras}. 
Você pode usar estilos de codificação para nomear seus arquivos, por exemplo. 
Dentre esses estilos destaco: \textsf{camelCase}; \textsf{CapitalCase}; 
\textsf{snake\_case}; \textsf{kebab-case}, ou uma mistura entre eles. 
Veja algumas possibilidades:

\tcboxC{
  \footnotesize
  \begin{tblr}{l}
    artigoNumerosComplexos.tex      \\
    ArtigoNumerosComplexos.tex      \\ 
    artigo\_numeros\_complexos.tex  \\
    artigo-numeros-complexos.tex    \\
    artigo\_numeros-complexos.tex
  \end{tblr}
}

Particularmente, prefiro a última opção: uma mistura entre \textsf{snake-case} 
e \textsf{kebab-case}. 

Há quem prefira escrever todo o texto num único arquivo \texttt{.tex}, 
inclusive as configurações. 
Entretanto, nesse minicurso, vamos separar as coisas\ldots 

\section{O arquivo mestre} %------------------------------------------------

Suponha que esse artigo hipotético sobre \textit{números complexos} seja 
composto por cinco partes: 
introdução histórica; 
representações dos números complexos; 
raízes de um número complexo; 
teorema fundamental da álgebra e a 
conclusão. 
Como há uma sequência subjacente, podemos enumerar esses itens, bem como usar 
a forma de escrita que aprendemos na seção anterior. 
A ideia é reduzir nomes, sem contudo deixar de lembrar o conteúdo principal de 
cada texto. 
Uma possibilidade é dada a seguir:

\tcboxC{
  \begin{tblr}{l}
    01\_intro-historica.tex; \\
    02\_representacoes.tex;  \\
    03\_raizes.tex;          \\
    04\_teo-fundamental.tex; \\
    05\_conclusao.tex        \\
  \end{tblr}
}

Então, como concatenar essas seções?
Fazemos isso por meio de uma função no \LaTeXX\ em um arquivo principal (ou mestre). 
Poderia ser feito no arquivo \texttt{artigo\_numeros-complexos.tex}, mas vamos 
denominá-lo simplesmente de \texttt{main.tex} (que quer dizer \textit{principal}). 
É um nome padrão: pequeno, sem acentuação e que traduz o que faz. 
Além disso, se você seguir essa idea em outros trabalhos, poderá configurar seu 
editor de \LaTeXX\ com atalhos convenientes. 

\begin{atencao}{Atenção!}{\exclamacao}
  \sffamily
  Nessa configuração, a compilação deve ser realizada no arquivo\; \texttt{main.tex}
\end{atencao}

Também faremos isso no \textsf{Overleaf}: criaremos um projeto e, dentro dele, 
organizaremos com diretórios distintos para arquivos distintos (\texttt{figs/}, 
para figuras; \texttt{tex/}, para arquivos \texttt{.tex}; etc.); e um arquivo por 
nome \texttt{main.tex}, onde faremos a compilação e onde estará comandos que 
buscam as seções do artigo no diretório \textit{tex/}.

Vamos conhecer a estrutura geral de um arquivo em {\grega \hologo{LaTeX2e}} 
para assimilarmos melhor essa ideia da organização. 

\section{{\grega \hologo{LaTeX2e}}: como começar a escrever} %-------- 

Sendo direto, uma estrutura global e minimalista para começar a escrever é:

\begin{codigo}{Estrutura global e mínima}{\lapis}
\documentclass{article}
\begin{document}
  Podemos escrever aqui!
\end{document}
\end{codigo}

Mas, o que significam essas coisas?

Uma das características mais forte do \LaTeXX\ é a possibilidade de separarmos 
o \textit{conteúdo} da \textit{formação}. 
Quando iniciante nessa linguagem, configurar um arquivo da forma como você quer, 
é um grande desafio. 
Mas, como existem uma infinidade de \textit{templates} que se adequam à maioria  
de nossas necessidades, basta inserirmos o \textit{conteúdo} e aprendermos algumas 
confugurações daquele \textit{template}. 

Agora, como organizar esse conteúdo? 

Bom\dots\ para escrever em \LaTeXE\ você precisa declarar a \textis{classe} e 
o \textis{ambiente} principal para escrita. 

No que se refere à \textit{classe} existem uma infinidade delas! 
Algumas já vem na distribuição básica do \LaTeXX, outras podemos instalar em 
nosso sistema. 
Dentre as classes \textit{standard} (padrão), destacamos: \textit{book}, para 
escrita de livros; \textit{report}, para escrita de relatórios; \textit{article}, 
para escrita de artigos. 
Cada uma dessas classes possuem características distintas e não conseguiremos cobrir 
nesse minicurso. 
Por isso, vamos nos ater \textsc{apenas} na classe \textit{article}. 

Como ``dizemos'' ao \LaTeXX\ que a classe que vamos trabalhar é a \textit{article}?
Simples: usamos o comando 

\tcboxC{
  \ttfamily
  \textbackslash documentclass\{article\}
}

Note que um comando no \LaTeXX\ começa (sempre) com uma ``barra invertida'', 
chamada de \textit{backslash}. 
Além disso, um comando pode possuir \textit{opções} ou \textit{argumentos}. 
Geralmente, as \textit{opções} são colocadas dentro de \textsf{colchetes}.
Só a nível de exemplificação, se colocarmos, no argumento do comando 
\verb|\documentclass}|, a classe \textit{article}; podemos, dentre outras opções, 
escolher:

\begin{enumerate}
  \item \textis{Tamanho da fonte}. Para todo o corpo do texto, há as opções: 
        \texttt{10pt}, \texttt{11pt} e \texttt{12pt}. Por padrão, vem \texttt{10pt}.
  \item \textis{Tipo do papel.} Folha A4: \texttt{a4paper}; Folha A5: \texttt{a5paper}; 
        etc. Por padrão, vem o americano: \texttt{letterpaper};
  \item \textis{Alinhamento da numeração em equações}. Por padrão, as equações,
        são enumeradas à direita, mas você pode mudar isso, colocando a opção 
        \texttt{leqno}, para alinhas à numeração à esquerda;
  \item \textis{Texto em duas colunas}. Para colocar texto em duas colunas, 
        usamos \texttt{twocolumn}.
  \item \textis{Modificação das margens}. Na classe \textit{article}, as margens 
        são configuradas para que a impressão seja em apenas um lado. 
        Mas, você pode colocar a opção \texttt{twoside} para modificar isso e 
        configurar para impressão em dois lados.\\ 
        \emoji{warning}\; Isso não diz pra sua impressora que ela tem que imprimir 
        em dois lados!! 
        Apenas configura as margens para que você possa configurar sua improssora 
        para o trabalho! 
\end{enumerate}

Para nossos fins, apenas usatemos a opção da fonte \texttt{12pt}. 
O tamanho do paper, vamos escolher em outro momento. 

No que se refere ao \textsf{ambiente princial para escrita}, o comando 
\verb|\begin{document}| foi indicado. 
Entre ele e o \verb|\end{document}|, escrevemos nosso texto! 

Comandos que possuem uma estrutura como:

\tcboxC{
  \textbackslash begin\{comando\}
  ...
  \textbackslash end\{comando\}
}
denominamos de \textis{ambientes}. 

Além disso, tudo o que bem antes do \verb|\begin{document}| até o \verb|\documentclass|, 
denominamos de \textis{preâmbulo}. 
No preâmbulo pode conter configurações para todo o texto. 
