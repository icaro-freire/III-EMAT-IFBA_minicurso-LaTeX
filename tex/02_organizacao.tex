%==========================================================================
% Capítulo 2: Organizando para Começar 
%==========================================================================
\chapter{Organizando para Começar} 
\label{cap:organizando} 


Esse capítulo poderia ser opcional, mas acredito que, depois de um tempo usando 
o \LaTeXX, você começará a pensar sobre \textit{organização}. 

E existem muitas vertentes do que seja ``organização'': podemos falar em como 
\textit{escrever} o nome dos arquivos que vamos usar para escrever em \LaTeXX;
podemos falar na organização do diretório onde faremos a composição tipográfica
(colocaremos tudo em um único diretório, ou vamos separar por categorias os 
arquivos?); podemos também pensar na organização da escrita do texto (vamos 
escrever tudo em um único arquivo, ou poderíamos escrever partes do texto em 
arquivos diferentes e, depois, juntarmos tudo num arquivo principal?); etc. 

É sobre isso que falaremos nesse capítulo. 
Mas, desde já, confesso que tudo será bem pessoal. 
Você pode fazer como desejar!

\section{Como escrever os nomes do arquivo para compilação} %---------------

Os compiladores são suficientemente inteligentes para identificar a linguagem 
{ \grega \hologo{LaTeX2e} } em um arquivo de texto. 
Mas, é uma boa prática salvar os arquivos no formato \texttt{.tex}. 
Isso facilita a identificação de ícones por parte das IDEs, por exemplo.

Também são boas práticas: salvar os arquivos \texttt{.tex} com nomes 
\textit{sem acentuação} ou \textit{caracteres especiais do teclado} (\%, \$, \&, etc.). 
E, se o nome do arquivo for composto por mais de uma palavra, é aconselhável 
também \textit{não deixar espaço entre elas}.

Por exemplo, suponha que você esteja escrevendo um \textit{artigo} sobre 
\textit{números complexos}. \textbs{Não} escreva assim: 

\tcboxC{
  Artigo! Números Complexos.tex
}

Seria aconselhável retirar a \textit{exclamação}, o \textit{acento agudo} e os 
\textit{espaços entre as palavras}. 
Você pode usar estilos de codificação para nomear seus arquivos, por exemplo. 
Dentre esses estilos destaco: \textsf{camelCase}; \textsf{CapitalCase}; 
\textsf{snake\_case}; \textsf{kebab-case}, ou uma mistura entre eles. 
Veja algumas possibilidades:

\tcboxC{
  \footnotesize
  \begin{tblr}{l}
    artigoNumerosComplexos.tex      \\
    ArtigoNumerosComplexos.tex      \\ 
    artigo\_numeros\_complexos.tex  \\
    artigo-numeros-complexos.tex    \\
    artigo\_numeros-complexos.tex
  \end{tblr}
}

Particularmente, prefiro a última opção: uma mistura entre \textsf{snake-case} 
e \textsf{kebab-case}. 

Há quem prefira escrever todo o texto num único arquivo \texttt{.tex}, 
inclusive as configurações. 
Entretanto, nesse minicurso, vamos separar as coisas\ldots 

\section{O arquivo mestre} %------------------------------------------------

Suponha que esse artigo hipotético sobre \textit{números complexos} seja 
composto por cinco partes: 
introdução histórica; 
representações dos números complexos; 
raízes de um número complexo; 
teorema fundamental da álgebra e a 
conclusão. 
Como há uma sequência subjacente, podemos enumerar esses itens, bem como usar 
a forma de escrita que aprendemos na seção anterior. 
A ideia é reduzir nomes, sem contudo deixar de lembrar o conteúdo principal de 
cada texto. 
Uma possibilidade é dada a seguir:

\tcboxC{
  \begin{tblr}{l}
    01\_intro-historica.tex; \\
    02\_representacoes.tex;  \\
    03\_raizes.tex;          \\
    04\_teo-fundamental.tex; \\
    05\_conclusao.tex        \\
  \end{tblr}
}

Então, como concatenar essas seções?
Fazemos isso por meio de uma função no \LaTeXX\ em um arquivo principal (ou mestre). 
Poderia ser \texttt{artigo\_numeros-complexos.tex}, mas vamos denominá-lo de 
\texttt{main.tex} (que quer dizer \textit{principal}). 
É um nome padrão: pequeno, sem acentuação e que traduz o que faz. 

\begin{atencao}{Atenção!}{\exclamacao}
  \sffamily
  Nessa configuração, a compilação deve ser realizada no arquivo\; \texttt{main.tex}
\end{atencao}
