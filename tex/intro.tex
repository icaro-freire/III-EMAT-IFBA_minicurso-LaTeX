%===========================================================================
% Introdução 
%===========================================================================

\chapter*{Bate-papo inicial} 
\label{cha:intro}

Alguns autores, quando começam a escrever o primeiro parágrafo de seus textos, 
trazem informações importantes, geralmente exibindo as cláusulas de um contrato
social informal entre ele e o leitor!
Isso prende a atenção do leitor instigando-o à leitura completa do texto\ldots\
Bom, mas, nesse texto as informações inportantes começam no segundo parágrafo. 
\emoji{smirking-face}

Bom\ldots\ se você continuou ler até aqui, devo confessar que esse texto não 
tem a pretenação de ser um manual, ou algo do tipo. 
Bem\ldots\ talvez algo do tipo. 
Ainda não sei precisar o bastante. 
A certeza que tenho é que não é humanamente possível explicar tudo, ou, pelo menos,
um mínimo aceitável sobre {\LaTeXX}\ em 3\unit{h}. 
Por isso o nome desse minicurso é `` {\fontins Tangenciando o \grega \LaTeX} ''. 
Abordaremos pontos parcialmente selecionados, ou seja, pontos que considero 
essenciais conhecer de início; e que, no meu entender, possa ser útil mesmo 
depois de algum tempo, dando uma direção para futuros estudos ou aprofundamentos. 

É uma tarefa difícil fazer essa seleção\ldots
Alguns vão achar que faltou falar sobre ``tal tópico''; outros vão dizer que 
o assunto abordado não é nada introdutório; etc. 
Tenham, então, em mente, que esse texto é bem parcial e reflete minha visão sobre 
o que  apresentar sobre o \LaTeXX\

O que pretendo abordar?

\begin{enumerate}
  \item \textbs{O que é} \textbf{\LaTeXX ?}
        Vou falar sobre a diferença entre compiladores; interpretadores; 
        distribuições; editores; e, principalmente, afirmarmos que \LaTeXX\ é 
        diferente de Látex.
  \item \textbs{Organizando para começar.}
        Nesse capítulo, apenas dou dicas de organização: como escrever nomes de 
        arquivos; como organizar um projeto de \LaTeXX; cuidados com os 
        diferentes tipos de fontes; a importância de versionar o projeto; 
        ferramentas para automação; etc.
  \item \textbs{Modo texto.}
        Aqui abordarei o básico sobre ênfase nas fontes; tamanhos; espaçamentos; 
        ambientes; etc.
  \item \textbs{Modo Matemático.}
        A cereja do bolo!
        Abordarei aspectos gerais, mas tentarei colocar muitos exemplos práticos 
        de um ``dia a dia'' de um professor de matemática.
  \item \textbs{Onde buscar ajuda.}
        Como eu já avisei que esse material não é um tutorial, vou mostrar os 
        tutoriais já consolidados; ferramentas para busca de símbolos; fóruns 
        voltados para dúvidas sobre \LaTeXX; canais do Telegram ou Discord; etc.
\end{enumerate}

Espero que gostem e ajudem vocês a escrever textos com qualidade tipográfica 
adequada e inspiradora. 

