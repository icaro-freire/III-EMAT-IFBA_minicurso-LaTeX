%===========================================================================
% Capítulo 1: O que é LaTeX?
%===========================================================================
\chapter{O que é \LaTeXX} 
\label{cap:LaTeX}

Esse capítulo é um pouco polêmico!
Alguns provavelmente falarão que não precisaria abordae tais tópicos em um 
minicurso de 3\unit{h}, vist que poderíamos ser mais diretos e realizarmos 
logo a prática. 
Entretanto, entendo que é um tópico importante, que definirá algumas ideias 
sobre o que é o \LaTeXX, bem como se propõe a uma abordagem mais atual dos 
recursos disponibilizados para composição tipográfica. 
Penso que, se eu soubesse dessas coisas logo na minha iniciação no \LaTeXX, 
teria uma visão bem mais consistente de seu uso. 

% seção 01 =================================================================

\section{Como se fala \LaTeXX?}
\label{sec:como-se-fala}

Pode até ser irrelevante essa informação, mas é sempre bom conhecermos as 
coisas corretamente, não? 

Sei que vocês sabem que não estamos falando daquela seiva da árvore que é 
usada para produzir diferentes materiais como tinta, luvas, etc. 
Ou seja, não estamos falando do Látex. 

O \LaTeXX\ não é Látex. 
Nem na ideia, nem na pronúncia. 

Existem duas maneiras aceitáveis para pronunciar o nome em questão: uma maisvoltada ao inglês e outra mais ``abrasileirada''. 

Antes de exibir a pronúncia, entretando, vamos conhecer a pronúncia da 
palavra \TeX.
Não é \textit{Téckis}, mas \textit{Téc} (como em ``\textbf{tec}nologia''). 
Isso se deve ao fato de que a palavra \TeX, na realidade, veio da junção 
das três letras gregas: {\grega τ} (tau);\, {\grega ε} (épsilon); e, 
{\grega χ} (chi --- lê-se \textit{Qui}, como em ``\textbf{Qui}abo'').
Essas letras gregas, juntas e em maiúsculas, ficam: {\grega ΤΕΧ}; mas o 
idealizador dessa linguagem de programação, como veremos, denominada \TeX, 
simplesmente quis modificar a disposição do {\grega Ε} para mostrar que se 
trata de algo relacionado à tipografia.

A tipografia é a arte e o processo de criação na composição e impressão de um texto, física ou digitalmente (Wikpédia).

É interessante notarmos que a palavra \TeX\ está relacionada à 
\textsf{tecnologia}, visto que é uma linguagem de programação; mas também 
relacionada à \textsf{arte}, visto que a tipografia é descentende da 
\textit{Caligrafia} --- uma arte que resiste ao tempo e traça muitos parâmetros para formas elegantes de letras/fontes.
Isso não é ao acaso!
A palavra \TeX\ foi escolhida justamente porque as palavras \textsf{arte} 
({\grega τεχνη}) e \textsf{tecnologia} ({\grega τεχνολογία}) possuem a mesma raiz linguística {\grega τεχ}. 

Do exposto, agora podemos expor as formas aceitáveis de pronunciar a palavra\LaTeXX.

\begin{itemize}
  \item[\emoji{speaking-head}] Se você quiser pronunciar mais parecido com o        Inglês, fale ``\textit{LeiTéc}''. 
  \item[\emoji{speaking-head}] Mas, se quiser falar mais abrasileirado, use 
        ``\textit{LaTéc}''.
  \item[\emoji{prohibited}] Apenas, evite, depois desse minicurso, falar 
        ``\textit{LaTéckis}''; ou, pior ainda, ``\textit{LáTeckis}''. \emoji{winking-face-with-tongue}
\end{itemize}




